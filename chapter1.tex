\doublespace
\Chapter{INTRODUCTION}
		\section{Motivation of the Thesis}
		When Sir Isaac Newton first postulated his laws of motion in the seventeenth century, he could not have anticipated that, hundreds of years later, these very same principles would
		be taught worldwide in high schools to teenagers as an introduction into the elementary concepts of physics. Nor could Newton have anticipated that in the twentieth century a
		physicist by the name of Albert Einstein would pioneer the concept of relativity, challenging the paradigm of Newton's laws which seemed so absolute in empirical practice. Further study
		since Einsten's time has revealed that this newer model for motion more closely aligns with the way in which the universe functions, while Newton's laws of motion do not hold
		under extreme conditions.

		And yet, it is Newton's model of classical physics that is often chosen to be taught to students as an introduction to physics despite these statements lacking a degree of truth when
		it comes to physical reality. Why is this the case when we have a model for the physical world that seems closer to the truth? The answer likely comes down to the \textit{efficacy} and
		\textit{complexity} of the models at hand. While Newton's model will predict inaccurate behavior in extreme conditions such as bodies moving at relativistic speeds, in the
		vast majority of applications it will give results that are in line with physical behavior while being far easier to understand, at least for an introductory physics student, than 
		Einstein's model. Of course, in more sophisticated applications than calculating the velocity of a ball rolling down a hill, one may want Einstein's model, but in this problem of 
		which model to select when instructing youth, Newton's model is sufficient.

		This example brings to mind the challenges statisticians face when performing statistical model selection. Model selection is the task of selecting a statistical model from among a
		collection of candidate models. One can use various techniques to perform model selection, but most will focus in some way on balancing \textit{goodness-of-fit} of the model with
		regards to a set of observations with \textit{parsimony} and \textit{interpretability}. These traits often go hand-in-hand with the two main objectives of model selection and 
		statistical modeling in general, those being statistical inference and prediction of future data. The priority in balancing these different aspects and objectives may change depending
		on the modeling problem at hand, similar to the physics example presented above where the simplicity of one model is preferred over the edge in aligning with empirical data of another.
		Thus, it is important to keep one's desired goals in mind when performing model selection, and to be aware of the different tendencies of various model selection procedures to favor
		one objective over the other.

		When performing statistical model selection, the use of likelihood-based information criteria is commonplace. Akaike pioneered the first such criterion
		with the resultant statistic coming to be known as the Akaike information criterion, or AIC for short (Akaike, 1974). AIC features a likelihood-based goodness-of-fit term and a
		penalty term to reflect the complexity of the model, thus providing a balance between these two aspects. Akaike derived AIC as an estimator to the expected Kullback-Leibler discrepancy,
		which is itself a relative measure of the separation between one probability distribution and another (Kullback and Leibler, 1951). The Kullback-Leibler discrepancy ignores constants
		associated only with the true generating distribution that are present in the Kullback-Leibler information from which the discrepancy is derived, causing AIC to have relative meaning among
		fitted candidate models, but little absolute meaning. In the case of application of AIC, the probability distributions of interest are those of the true generating model
		for our data and our proposed model that has been fit using the data. The lower the value of AIC for a fitted model, the lower our estimate for the Kullback-Leibler divergence,
		and thus the lower the separation between the true generating model and fitted candidate model. Thus, a recommended practice has been to perform model selection by choosing
		the candidate model with the lowest value of AIC.

		Over the course of time, others have produced refinements to Akaike's work either by producing different likelihood-based information criteria, or modifying the ways in which AIC or other
		criteria are used to select a model. The Bayesian information criterion (BIC) features a Bayesian justification for its usage as opposed to one of discrepancy estimation, but is similar to AIC in that
		it features a term based on the likelihood and a term that penalizes greater model complexity (Schwarz, 1978). AIC itself is only an asymptotically unbiased estimator of the expected Kullback-Leibler discrepancy,
		and thus the corrected Akaike information criterion (AICc) was developed to serve as an exactly unbiased estimator in the case of linear regression (Sugiura, 1978).
		Both BIC and AICc provide potential advantages over AIC, and share the property with AIC that lower values of the criterion indicate more support for a model. Thus, one can select a candidate model
		from a collection by choosing the model with minimum BIC or AICc, similar to to AIC.

		Some have suggested more sophisticated methods of using likelihood-based information criteria than simply choosing the model with the lowest value of the criterion. In their 2003 work on 
		model selection, Burnham and Anderson offer the following table as a guide when assessing differences in AIC values between the model with the minimum AIC and another model:
		\pagebreak

		\begin{table}[h]
		\centering
		\ttabbox[\FBwidth]
		{\caption{\label{tab:Burnham_Anderson_Table}Burnham and Anderson AIC Difference Reference Table}}
		{
		\begin{tabular}{ c|c}
		$AIC_{i}-AIC_{min}$ & Level of Empirical Support for Model $i$\\
		 \hline
		 $0 - 2$ & Substantial\\
		 $4 - 7$ & Considerably Less\\
		 $> 10$ & Essentially None\\
		 \Xhline{3\arrayrulewidth}
		\end{tabular}
		}
		\end{table}	
		
		This table contends that any model with an AIC difference of less than $2$ from the minimum AIC model is a model for which the data shows substantial support. Thus, this ``Rule of 2" has become a
		commonplace rule of thumb when using AIC for model selection, with 2 often being used as the cutoff to dictate a meaningful difference in model fit. The justification for this cutoff originates
		from empirical observations (Burnham and Anderson, 2003). Additionally, different  arguments have been made for using 2 as a cutoff for a meaningful difference in model fit
		when comparing BIC values (Kass and Raftery, 1995). However, these cutoffs suffer from thin justification in terms of the the distributional properties of the likelihood-based information
		criteria in question.

		If one wishes to perform model selection that relies on distributional properties, then the likelihood ratio test may be of interest. The procedure tests the assumption that
		a smaller, nested model fits the data just as well as a larger model with higher dimensionality, and additionally carries an implicit assumption that the larger model does
		not exhibit substantial lack of fit itself (Wilks, 1938). Thus, one must make some effort to ensure that this larger candidate model does not exhibit gross lack of fit.

		Various methods exist to assess misspecification of a model, with most being tuned to a specific family of models or aspect of misspecification. In the case of linear regression,
		plotting the fitted residuals against the fitted values or individual covariates is a common form of visually assessing whether the linear regression assumptions of linearity and
		homoskedasticity are met (Kutner et al., 2005). In addition to visual diagnostics, several hypothesis testing methods have been developed to assess the goodness-of-fit of linear
		models. Methods such as the Breusch-Pagan test (Breusch and Pagan, 1979) and the White test (White, 1980) can be employed to test the hypothesis that the residuals exhibit homoskedasticity
		after the data is fit to a linear model. These methods excel at detecting certain forms of misspecification in linear models, but can be fooled when the misspecification manifests in a
		way they are not tuned to detect, and are not robust to all violations of the assumptions of linear regression.

		The main motivation for this thesis is to investigate the distributional properties of likelihood-based information criteria with the hope of producing methods to aid in model selection
		that rely on probabilistic theory as opposed to ad hoc rules of thumb and visual inspection. The first part of the novel developments in this thesis outlines a new model selection procedure using the distributional
		results of the likelihood ratio test by applying these to a difference in likelihood-based information criteria. The method can be used with any likelihood-based information criterion
		with a non-random penalty term, and quantities related to its usage with AIC, BIC, and AICc will be explicitly derived. A simulation study follows which demonstrates that this algorithm
		can provide an improvement in selection of the true model and predictive efficacy over other model selection procedures using the same information criterion.
		
		In the second portion of the thesis, distributional properties related to the large-sample distribution of the goodness-of-fit term present in likelihood-based information criteria
		are established. These properties are then leveraged to create a bootstrap goodness-of-fit test to test the very general hypothesis that a linear regression model is properly specified
		against the alternative that it is not. The ability of this test to perform as expected when the null hypothesis is true is verified through a simulation study. Further simulation
		studies display the advantage of this test over existing methods as it pertains to the ability to detect subtle, but notable, forms of misspecification.

		\section{Overview of the Thesis}
		The remainder of the thesis is organized as follows. In Chapter~2, we provide some background related to likelihood-based information criteria, goodness-of-fit assessment, and other mathematical
		constructs that will be of relevance later in the dissertation. In Chapter~3, we propose a new model selection procedure using likelihood-based information criteria that leverages distributional
		properties. We investigate the performance of the procedure as compared to other methods in illustrative simulation studies.  In Chapter~4, we derive quantities related to the variance of the
		log-likelihood term used in information criteria, and we use these findings to develop a bootstrap test for the goodness-of-fit of a linear model. Simulation studies follow to demonstrate the
		efficacy of the developed procedure. In Chapter~5, we present an analysis using real world data to display the methods developed in this thesis in an applied setting.
		Finally, Chapter~6 concludes with a brief summary and discussion of the content of the thesis. 
		