\doublespace
\Chapter{UNIFIED APPLICATION OF DEVELOPED PROCEDURES}

		This thesis has presented a new method for using likelihood-based information criteria to select from a collection of candidate models, and a new procedure
		for testing the goodness-of-fit of a fitted normal linear regression model. Both of these developments will be used in this section in a brief application
		involving real-world data to demonstrate how the methods can be used in practice.

		\section{Application Formulation} \label{sec:app_form}

		We begin by defining the data that will be used in this application. Say an investigator wishes to explore the relationship between cancer-related mortality in
		the United States at the county level and various household-related variables. The data used in this example was aggregated from the National Cancer Institute
		and the U.S. Census Bureau, with the data in question being being collected in the year 2013.
		
		The primary outcome of interest will be cancer mortalities per 100,000 county residents, with data being collected for 300 U.S. counties. Additionally, the study has access to the
		following covariates for each county: cancer diagnoses per 100,000 county residents, median household income in the county in tens of thousands of dollars, median age of county residents, mean
		household size of county, percent of households in the county that are in poverty, percent of county residents who are married, and lastly the birth rate of the county expressed
		as the number of live births in the year 2013 relative to the number of the women in the county.

		We wish to build a statistical model to explore which household-related covariates are related to mortality at the county level. We will require that any model we
		consider include cancer diagnoses and age as covariates, as we suspect these will be highly related to cancer mortality and must therefore have their effects
		accounted for in the model. All other covariates will be candidates for selection.

		The analysis will commence as follows. Initially, a normal linear regression model with cancer mortality as the outcome, and with an intercept term in addition to all
		other available variables as covariates, will be fit and considered. The bootstrap goodness-of-fit procedure presented in Chapter 4 will be applied to test the 
		viability of the proposed linear regression model. If the test rejects that the model is correctly specified at the $\alpha = 0.05$ level, other modeling frameworks
		will be explored. However, if the null hypothesis is not rejected, we proceed as if the proposed normal linear regression model does not exhibit clear lack-of-fit.

		Assuming we have a largest candidate model in hand that does not exhibit lack-of-fit, this model will be used as the basis for model selection using the
		AIC-based procedure presented in Chapter 3. The candidate models will be all models that contain an intercept term, an effect for age, an effect for cancer
		diagnoses, and an effect for at least one other household-related covariate. This will include the largest candidate model, and a group of other models that
		are nested within this model, creating the setup needed to use the standardized AIC model selection algorithm. The procedure will be performed and a final model
		will be chosen, with a discussion of the covariates chosen to be in the model to follow. This analysis will be performed R, version 4.2.1 (R Core Team, 2023).
		
		\section{Application Results} \label{sec:app_res}

		To begin, the specified candidate regression model with cancer mortality as the outcome and all other covariates available as effects was fit to the data. A
		figure depicting a residual plot for the fitted model can be found below.

		\begin{figure}[H]
			\centering
			\captionsetup{justification=centering}
			\includegraphics[width=1\textwidth]{figures/application_residual_plot.pdf}
			\caption{\label{fig:app_residual_plot} Application Largest Candidate Model Residual Plot}
		\end{figure}

		This residual plot does not appear to reveal any lack-of-fit in this model. However, it has been noted earlier in this thesis that this method of visual
		inspection can mask certain forms of misspecification. Thus, the bootstrap goodness-of-fit test was employed on this largest candidate model to assess
		goodness-of-fit. The test was performed at the level $\alpha = 0.05$ with $B = 1000$ bootstrap iterations. The generated bootstrap confidence interval
		was found to be
		\begin{equation*}
			\left[ 578.9217, 974.5816 \right].
		\end{equation*}
		Note that the value under the null hypothesis for the variance of the likelihood goodness-of-fit term in this scenario is $2n = 600$. Thus, as this interval contains the
		value of $600$, we would fail to reject the null hypothesis that this largest candidate linear model is properly specified, and are justified in proceeding
		under the assumption that the model does not show lack-of-fit.

		With this in mind, we proceed to selecting a final model. All candidate models were fit to the data, and the standardized AIC of each model was calculated
		using the established largest candidate model as a reference. The following plot displays each standardized AIC of each candidate model versus the difference in
		degrees of freedom between the candidate model and the largest candidate model. The horizontal purple line signifies the cutoff line of $\sqrt{\frac{1}{2}} + 2$.

		\begin{figure}[H]
			\centering
			\captionsetup{justification=centering}
			\includegraphics[width=1\textwidth]{figures/application_plot_uncolored.pdf}
			\caption{\label{fig:app_stand_AIC_plot_no_colors} Standardized AIC vs. Difference in Degrees of Freedom from Largest Candidate}
		\end{figure}

		All models below the cutoff line are candidates for selection, with models that are below this line and farther to the right to be preferred due to parsimony. Note
		that there is a single model that appears to have both the maximal value of difference in degrees of freedom, and is below the cutoff. This model is the
		final selected model, and contains only an effect for median income within a county, in addition to the forced intercept term, term for county cancer incidence,
		and term for median age in the county.
		
		The effect for median income was found to be negative, indicating that as median income of a county rises, the cancer mortality per capita may tend to
		decrease.  Interpreting the regression coefficient $-7.313$, one could say that as the median income of a county rises ten-thousand dollars with all other factors being
		held equal, we would expect the cancer mortality per 100,000 residents of the county to decrease by $7.313$. This aligns with intuition, as it seems reasonable
		that more affluent communities will likely have access to better medical care. However, one might be skeptical that this is the only covariate available with a meaningful
		relation to cancer mortality, as similar arguments could postulate the importance of other household covariates. Along this line of thinking, the following plot
		is presented. This plot is similar to Figure 5.2, but now color codes models based on whether they included the variable for a county's median household income.

		\begin{figure}[H]
			\centering
			\captionsetup{justification=centering}
			\includegraphics[width=1\textwidth]{figures/application_plot_colors.pdf}
			\caption{\label{fig:app_stand_AIC_plot_colors} Standardized AIC vs. Difference in Degrees of Freedom from Largest Candidate,
			Color-Coded by Inclusion of Median Income Covariate}
		\end{figure}

		Note that all models below the cutoff line, which are to be treated as serious candidates for selection, contain the median income covariate, while all models
		above the cutoff, which display lack-of-fit, do not. The inclusion vs. exclusion of this variable drastically alters the validity of each model as a whole, and
		thus we can say that it appears this single covariate may have a meaningful relationship with cancer mortality at the county level. This negative association between
		median income and cancer mortality at the county level underscores the importance of work related to socioeconomic disparities in health and healthcare in the United States,
		and motivates future work to investigate this relationship and inform possible future policy.

