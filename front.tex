% This file specifies information specific to your thesis, such as your
% title, advisor, dedication, etc.


% To remove optional components, comment out the line
\titlepgtrue
\copyrighttrue %(optional)
%\signaturepagetrue %No longer needed since committee signature is now done virtually
\acktrue %(optional)
\dedicationtrue
\tablecontentstrue
\tablespagetrue
\figurespagetrue

\title{Probability Distribution Informed Extensions \\
for Classical Model Selection Methods:\\
Moving Beyond the ``Rule of 2''}
\author{Scott H. Koeneman}
\advisor{Professor Joseph Cavanaugh}
\dept{Biostatistics}
\submitdate{April 2023}
\supervisor{Joseph Cavanaugh}
\membera{Aaron Miller}
\memberb{Gideon Zamba}
\memberc{Jacob Oleson}

\newcommand{\abstextwithesis}
{
The use of likelihood-based information criteria such as the Akaike information criterion (AIC), Bayesian information criterion (BIC),
and corrected Akaike information criterion (AICc) for model selection is commonplace. When employing such criteria to compare candidate models,
lower values of a criterion are typically indicative of models that achieve an optimal balance between fidelity to the data and parsimony. Model 
selection may be performed by choosing the model from among those under consideration with the minimum value of the information
criterion of choice. However, this method is not guaranteed to produce optimal results in finite sample sizes.

Another principle in using likelihood-based information criteria for model selection is often dubbed the ``Rule of 2." This rule
of thumb recommends that one should not discount models that possess a criterion value within 2 units of the minimum criterion value among
all candidate models, as opposed to simply choosing the model corresponding to the minimum criterion value. This rule can be shown to improve the performance
of model selection using information criteria in finite sample size applications, if first used to identify models that are serious candidates for
selection followed by selection of one of these models using some other convention.

However, while effective, the ``Rule of 2" possesses limited theoretical backing. Its justification largely lies on empirical observations
in a limited number of modeling scenarios and does not rely on probability theory or the distributions of the quantities at hand.

In this thesis, the distributional properties of likelihood-based information criteria are explored. These properties are then leveraged
to produce a new model selection procedure involving information criteria that is similar in spirit to the ``Rule of 2," but is backed
by more formal probability theory. Through simulations, this procedure is shown to be efficacious in both descriptive and predictive model selection
contexts as compared to other methods using the same information criterion. Additionally, a novel goodness-of-fit test
for normal linear regression models is developed using the distributional properties of information criteria, with further simulations
displaying the utility of this test. Finally, the model selection procedure and goodness-of-fit test are synthesized in an application
to real data, demonstrating how both can be used in practice.

}

\newcommand{\pubabstextwithesis}
{
Model selection is a task often undertaken by statisticians and non-statisticians alike. The goal of model selection may be prediction of future
outcomes, description of a data-generating process, exploration of a relationship within the data, or some combination thereof. To serve these goals,
various methods of performing model selection have been developed in the past.

One such method of model selection involves the use of likelihood-based information criteria such as the Akaike information criterion (AIC), Bayesian information
criterion (BIC), and corrected Akaike information criterion (AICc). These criteria help one
to identify a model that offers the best balance of conformity to the data and complexity, and can be used in a variety of different ways. A
rule of thumb for using likelihood-based information criteria is the ``Rule of 2," which prescribes that one should not discount models that possess
a criterion value within 2 units of the minimum criterion value among all candidate models. When applying this rule, one can often
achieve good efficacy in model selection.

However, this ``Rule of 2" possesses limited theoretical backing. In this thesis, the distributional properties of likelihood-based information criteria
are investigated with the goal of producing methods of model selection using likelihood-based information criteria that have sound backing in probability
and distribution theory. Using distributional results, a new model selection algorithm to select a model from a collection of candidates is presented, along
with a novel goodness-of-fit test for linear models. The efficacy of these methods is demonstrated in simulation studies, and additionally, an example
application involving public health data is presented.
}

\newcommand{\acknowledgement}
{
Firstly, I would like to acknowledge my advisor, Dr. Joe Cavanaugh. Joe has been with me every step of the way in my graduate school journey, and the ideas
and work in this thesis are as much attributable to him as they are to me. I couldn't have done this without you, Joe. Your mind for statistics, as
well as your kindness and friendship, have been a blessing that I do not deserve.

I would like to thank the University of Iowa, and specifically the Department of Biostatistics, for taking a chance on me and allowing me to study here these
past five years. I will always be grateful for the gift that was given me to pursue my PhD at Iowa. I would also like to thank all of my professors and classmates
who have supported me along the way, from classmates to committee members to fellow RAs to research collaborators. I will miss the department and my research
groups greatly.

Thank you to my family for supporting me all of these years. Mom, you told me long ago that, just as Spiderman's great power came with great responsibility, so too
do my gifts come with a responsibility. Thank you for encouraging me to reach higher. Thank you also for loving me so unconditionally. I am truly blessed to have a
mom like you. To my six brothers and sisters, thank you for being my oldest friends. We've stuck together this long, and I hope we never stop doing so.

Without my wife Arianna, I surely would have given up on this thesis at multiple points. Arianna, thank you for being by my side throughout this entire journey,
and I am beyond blessed to have a wife like you. 

Lastly, I would like to thank God, and His Son Jesus Christ, who died for the sins of the world, and conquered death in rising from the dead, so that we could be reconciled
to the Father. Apart from faith in what God has done, I have no hope in this world or the next, and I thank Him for graciously giving me so much that I do not deserve. If
you are reading this now, I invite you to consider this message for yourself, that you may be reconciled to a God who desires to know you through trusting in Jesus Christ. I truly
believe this message, and if you do not consider it on anything else, perhaps consider it based on my own belief that this thesis would be impossible without Him.
}

\newcommand{\dedication}
{
\begin{center}
\textit{To my mother and siblings,\\
for putting up with me for many years.}
\end{center}
}

\beforepreface
\afterpreface
