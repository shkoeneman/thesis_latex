% This file specifies information specific to your thesis, such as your
% title, advisor, dedication, etc.


% To remove optional components, comment out the line
\titlepgtrue
\copyrighttrue %(optional)
\signaturepagetrue
\acktrue %(optional)
\dedicationtrue
\tablecontentstrue
\tablespagetrue
\figurespagetrue

\title{Probability Distribution Informed Extensions for Classical Model Selection Methods: Moving Beyond the ``Rule of 2''}
\author{Scott H. Koeneman}
\advisor{Professor Joseph Cavanaugh}
\dept{Biostatistics}
\submitdate{April 2023}
\supervisor{Joseph Cavanaugh}
\membera{Aaron Miller}
\memberb{Gideon Zamba}
\memberc{Jacob Oleson}

\newcommand{\abstextwithesis}
{
Due to its generality and flexibility, the state-space model has become one of the most popular models in modern time domain analysis for the
description and prediction of time series data.  The model is often used to characterize processes that can be conceptualized as ``signal plus noise,"
 where the realized series is viewed as the manifestation of a latent signal that has been corrupted by observation noise.

In the state-space framework, parameter estimation is generally accomplished by maximizing the innovations Gaussian log-likelihood. The maximum likelihood
estimator (MLE) is efficient when the normality assumption is satisfied. However, in the presence of contamination, the MLE suffers from a lack of
robustness. Basu, Harris, Hjort, and Jones (1998) introduced a discrepancy measure (BHHJ) with a non-negative tuning parameter that regulates the trade-off
between robustness and efficiency. In this manuscript, we propose a new parameter estimation procedure based on the BHHJ discrepancy for fitting
state-space models. As the tuning parameter is increased, the estimation procedure becomes more robust but less efficient. We investigate the performance
of the procedure in an illustrative simulation study. In addition, we propose a numerical method to approximate the asymptotic variance of the estimator,
and we provide an approach for choosing an appropriate tuning parameter in practice.  We justify these procedures theoretically and investigate their
efficacy in simulation studies.

Based on the proposed parameter estimation procedure, we then develop a new model selection criterion in the state-space framework.
The traditional Akaike information criterion (AIC), where the goodness-of-fit is assessed by the empirical log-likelihood, is not robust to outliers.
Our new criterion is comprised of a goodness-of-fit term based on the empirical BHHJ discrepancy, and a penalty term based on both the
tuning parameter and the dimension of the candidate model.
We present a comprehensive simulation study to investigate the performance of the new criterion.
In instances where the time series data is contaminated, our proposed model selection criterion is shown to perform favorably relative to AIC.

Lastly, using the BHHJ discrepancy based on the chosen tuning parameter, we propose two versions of an influence diagnostic in the state-space framework.
Specifically, our diagnostics help to identify cases that influence the recovery of the latent signal, thereby providing initial guidance and insight for
further exploration. We illustrate the behavior of these measures in a simulation study.
}

\newcommand{\pubabstextwithesis}
{
Time series data are very common in many disciplines, including finance, economics, meteorology, ecology, and epidemiology. Investigators are often
interested in building statistical models to understand the mechanisms behind time series data, and to make forecasts to guide future decisions. Due
to its generality and flexibility, the state-space model is one of the most popular models in modern time series analysis.

Once a model is formulated, the parameters of the model must be estimated from the data. Two salient issues in parameter estimation include the
efficiency and robustness of the estimator. Efficiency refers to the principle of obtaining estimators with low variability based on the
size of the sample at hand; robustness pertains to the extent to which the estimator is likely to be influenced by outlying values. We develop
a parameter estimation procedure for fitting state-space models that balances efficiency and robustness. In addition, based on the estimation
procedure, we present a model selection criterion in the state-space framework that is robust to outliers. Finally, we propose two diagnostics
for state-space modeling to facilitate the identification of points that could substantially influence predictors generated by the fitted model.
}

\newcommand{\acknowledgement}
{
Placeholder
}

\newcommand{\dedication}
{
\begin{center}
\textit{To my mom,\\
for your endless support and love.}
\end{center}
}

\beforepreface
\afterpreface
