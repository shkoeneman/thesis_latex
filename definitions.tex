\def \thetanot{\theta_0}
\def \thetahat{\hat{\theta}}

\def \cYhat{c(Y,\thetahat)}
\def \cyhat{c(y,\thetahat)}
\def \clooYhat{c_{L1O}(Y,\thetahat)}
\def \clooyhat{c_{L1O}(y,\thetahat)}

\def \cZ{c(Z,\theta)}
\def \cz{c(z,\theta)}
\def \cZhat{c(Z,\thetahat)}
\def \czhat{c(z,\thetahat)}
\def \cZbarhat{\bar{c}_Z(\hat{\theta})}
\def \czbarhat{\bar{c}_z(\hat{\theta})}
\def \cZnot{c(Z,\thetanot)}
\def \cznot{c(z,\thetanot)}

\newcommand{\indep}{\rotatebox[origin=c]{90}{$\models$}} 
\newcommand{\rff}{\singlespace \vskip 0in\par\sloppy\hangindent=2pc%
                 \hangafter=1 \noindent }
\newcommand{\overbar}[1]{\mkern 1.5mu\overline{\mkern-1.5mu#1\mkern-1.5mu}\mkern 1.5mu}
\newcommand*\circled[1]{\tikz[baseline=(char.base)]{
            \node[shape=circle,draw,inner sep=2pt] (char) {#1};}}







%Below are Nan's definitions
\newtheorem{theorem}{Theorem}
\newtheorem*{theorem*}{Theorem}
\newtheorem{lemma}{Lemma}
\newtheorem*{lemma*}{Lemma}
\newcommand\numberthis{\addtocounter{equation}{1}\tag{\theequation}}
\newcolumntype{P}[1]{>{\centering\arraybackslash}p{#1}}
\def \hm{\widehat{\mu}}
\def \hs{\widehat{\sigma}}