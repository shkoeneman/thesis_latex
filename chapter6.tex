\doublespace
\Chapter{DISCUSSION AND CONCLUSIONS}

		In this thesis, a new method of model selection using likelihood-based information criteria was developed in addition to a new goodness-of-fit test
		to be employed for normal linear regression models. This chapter features a summary and discussion of these developments followed by a section
		detailing potential future developments. 

		\section{Summary and Discussion}

		The ``Rule of 2" has been widely used as a rule of thumb in conjunction with likelihood-based information criteria such as AIC and BIC. Despite the
		demonstrated efficacy of this rule, it possesses little theoretical backing and was largely justified by empirical results. One of the stated
		goals of this thesis was to explore the distributional properties of likelihood-based information criteria as a means to form alternative methods of
		model selection that possess firmer justification.
		
		Using distributional properties of the likelihood ratio test, Chapter 3 developed a new model selection procedure that standardizes the variance of
		a difference in criterion values between the largest candidate model and another nested candidate model. The property of unit variance of this
		standardized quantity will hold under the assumption that the nested model contains requisite structure with regards to the the true model, with an additional
		implicit assumption that the largest candidate model itself does not display gross lack-of-fit. Using this standardized quantity, an acceptable
		cutoff for feasibility of a model was formulated using further distributional theory with final selection of a model being done using the principle
		of parsimony.

		Various iterations of this procedure involving different model selection criteria were shown in a simulation study to be able to improve on the descriptive
		and predictive efficacy of other methods of model selection involving those same information criteria.  Additionally, the method was demonstrated to
		be able to be used both for variable selection and correlation structure selection. Thus, the task to develop an effective new method of model selection
		involving likelihood-based information criteria with more theoretical backing was accomplished.

		To complement this new model selection procedure, Chapter 4 developed a novel goodness-of-fit procedure to be used with normal linear regression models.
		The procedure involves findings related to the variance of the goodness-of-fit term for likelihood-based information criteria, and leverages these
		results to produce a bootstrap hypothesis test with the null hypothesis that a fitted normal linear regression model is properly specified. The alternative
		hypothesis for this test is that the fitted model is misspecified in some way.

		Simulation results showed that this goodness-of-fit test is roughly able to maintain the desired Type I error rate when the null hypothesis is true, subject
		to possession of a sufficient sample size. Simulation results also demonstrated that this test can detect forms of misspecification that other common
		goodness-of-fit tests cannot, such as mean misspecification or heteroskedasticity on account of an unobserved variable. Thus, the test was shown to be of use
		when assessing the appropriateness of a given linear regression model.

		The model selection procedure and goodness-of-fit test outlined above were synthesized in Chapter 5 in an example application using real data. A proposed
		largest candidate linear model was first validated as a reasonable modeling option using the bootstrap goodness-of-fit test. Following this, the new model selection
		procedure was employed to choose a best model from among the largest model and a collection of nested candidates. This application displays how the
		two developments can be employed together for a holistic approach to model selection.

		\section{Future Work}

		The developments presented in this thesis offer several avenues for future work and improvement. One such extension touched upon earlier in the thesis is the
		potential for the model selection algorithm to be applied to other likelihood-based information criteria beyond what was presented in Chapter 3. The extension
		will be straightforward in the case of a criterion with a constant penalty term. However, applying the algorithm to be used with a criterion with a non-constant penalty term,
		such as is the case for TIC, would likely require further innovation.

		Additionally, further investigation could be done to observe how the model selection algorithm performs when its initial assumption, that being the appropriate
		specification of the largest candidate model, is violated. The justification for AIC itself assumes that the true model is of the same parametric class as the candidate model
		at hand, but AIC has been shown to be relatively robust even when this assumption does not hold. Further work could be done to explore how the model selection
		framework presented in this thesis performs in various scenarios where the largest candidate model is misspecified in some fashion. Alternatively, modifications
		to the procedure could be sought that seek to allow this assumption to be relaxed.

		With regards to the goodness-of-fit procedure developed in this thesis, a natural path for further development may be to develop similar such procedures that can
		be used to assess models beyond the normal linear regression setting. The distributional quantities related to the variance of the likelihood goodness-of-fit term that were derived
		in this thesis only apply to the setting of normal linear regression. However, similar distributional properties could be explored for other modeling frameworks such
		as logistic regression. With these quantities in hand, a similar bootstrap goodness-of-fit test pertaining to these other modeling frameworks may be able to be constructed.
		
		The future prospects for extensions of the work in this thesis expand beyond what has been presented above, but these ideas represent a sample of further work that
		could be pursued.
		%While the methods introduced in this thesis show merit in and of themselves, an attitude of contentment with what has already been accomplished
		%would have precluded the development of these methods from the outset. Thus, further improvements in the realm of model selection must continue to be sought for as long as
		%man seeks to better quantify and predict the world in which we live.
		
		
		