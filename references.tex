\addtocontents{toc}{\protect\vspace{\li}}
\addcontentsline{toc}{head}{REFERENCES}
\bibliography{REFERENCES}
\begin{center}
\textbf{REFERENCES}
\end{center}
\singlespace

\phantom{a}

\phantom{a}

\rff Akaike, H. (1973),
     Information theory and an extension of the maximum likelihood principle, in: B. N. Petrov and F. Csaki, (Eds.).
     {\it 2nd International Symposium on Information Theory}
     (Akademia Kiado, Budapest), {267--281}.

\phantom{a}

\rff Burnham, K.P. and Anderson, D.R. (2003),
      Model Selection and Multimodel Inference: A Practical Information-Theoretic Approach.
      {\it Springer}.

\phantom{a}

\rff Hurvich, C.M. and Tsai, C.L. (1989),
      Regression and time series model selection in small samples.
      {\it Biometrika}.
      {\bf 76}, {297--307}.

\phantom{a}

\rff Kullback, S. and Leibler, R. A. (1951),
     On information and sufficiency.
     {\it The Annals of Mathematical Statistics}
     {\bf 22}, {79--86}.

\phantom{a}

\rff Schwarz, G.E. (1978),
      Estimating the dimension of a model.
      {\it Annals of Statistics}
      {\bf 22}, {461--464}.

\phantom{a}

\rff Sugiura, N. (1978),
      Further analysis of the data by Akaike's information criterion and the finite corrections.
      {\it Communications in Statistics - Theory and Methods}
      {\bf 7}, {13--26}.