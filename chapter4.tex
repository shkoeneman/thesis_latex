\doublespace
\Chapter{GOODNESS-OF-FIT PROCEDURE FOR NORMAL LINEAR REGRESSION MODELS}

	The preceding chapter presented a new model selection procedure that relies on the distributional properties of differences in information criterion values to guide
	selection in a manner that does not rely only on rules of thumb. However, for such properties to be justified, it was first required that the largest candidate model
	in a collection of candidate models does not display lack-of-fit. This section will develop a new goodness-of-fit procedure to be used with normal linear regression
	that can be employed to help justify such an assumption.
		
		\section{Asymptotic Variance of Log-Likelihood GOF Term} \label{sec:asymp_var}

		We will now explore the variance of the log-likelihood goodness-of-fit term present in likelihood-based information criteria. We will assume that we are in a
		scenario where a normal linear regression model is being fit to the data of interest, and that this model is not misspecified. Thus, this model is of the proper
		parametric family and contains the requisite mean structure, although the mean structure may contain extraneous variables in the
		case of an overspecified model.

		Assuming a linear model has been fit using maximum likelihood with fitted parameters $\hat{\theta}$, the goodness-of-fit term can be decomposed as
		\begin{equation}
			-2 \ell (\hat{\theta}  ) = n \log(2 \pi) + n + n \log(\hat{\sigma}^2 ) ,
		\end{equation}
		where $\hat{\sigma}^2$ denotes the maximum likelihood estimate for the error variance $\sigma^2$. Note that the only term here that is random is
		the $n \log(\hat{\sigma}^2)$ term. Thus, if we can quantify the variability of this term, we can quantify the variability of the entire goodness-of-fit
		term. Additionally, we can go even further and quantify the variability of an entire likelihood-based information criterion with a constant penalty
		term for this fitted model.

		To achieve this end, we first consider $\hat{\sigma}^2$. As the linear model is assumed to not be misspecified, and $\hat{\sigma}^2$ is a maximum likelihood
		estimator, this estimator will have an asymptotic variance related to the inverse of the Fisher information (Fisher, 1922). In the case of a single
		observation, the Fisher information as it relates to the parameter vector $\theta' = [\beta', \sigma^2]$, where $\beta$ represents the regression coefficients
		present in the model, can be shown to be
		\begin{equation*}
			- E \left[ \frac{\partial^2 \ell (\hat{\theta}  )}{\partial \theta^2} \right] = \mathcal{I}_{n}(\theta) =
			\begin{bmatrix}
				\frac{X' X}{\sigma^2} & 0 \\
				0 & \frac{n}{2 \sigma^4} \\
			\end{bmatrix}
			,
		\end{equation*}
		where $X$ is the design matrix of the regression, and $0$ is a vector of zeroes. Thus, if we assume we have a single observation
		such that $n=1$,we may take the inverse of the Fisher information matrix and isolate the element related to the error variance $\sigma^2$. We see that
		this element will be
		\begin{equation*}
			\mathcal{I}_{1}^{-1}(\sigma ^2) = 2 \sigma ^4 .
		\end{equation*}
		
		Using the above relation, and applying the property of asymptotic normality of the MLE $\hat{\sigma}^2$ in this case of a properly specified normal linear
		regression model, we see that the asymptotic distribution
		\begin{equation*}
			\sqrt{n} (\hat{\sigma}^2 - \sigma^2) \xrightarrow[]{d} N(0, 2 \sigma ^4 )
		\end{equation*}
		should hold. Thus, we have established an asymptotic variance for $\hat{\sigma}^2$.

		However, further work is required to arrive at what we seek. To find the variance of $-2 \ell (\hat{\theta}  )$, we must find the variance of $n \log(\hat{\sigma}^2)$.
		Additionally, the above asymptotic relation involves the true $\sigma^2$ to which we will not have access in most modeling scenarios.

		We will solve both of these issues by employing the delta method (Rao, 1965). We propose a transformation of the form
		\begin{equation*}
			g(x) = \log(x) .
		\end{equation*}
		Note that this transformation has the derivative
		\begin{equation*}
			g'(x) = \frac{1}{x} .
		\end{equation*}
		Thus, applying the delta method to our above asymptotic distribution with $g(x)$ as the function of interest, we see that
		\begin{equation*}
			\sqrt{n} ( \log (\hat{\sigma}^2) - \log(\sigma^2)) \xrightarrow[]{d} N(0, 2) .
		\end{equation*}
		Note that the asymptotic distribution on the right-hand side is now completely free of the parameter $\sigma^2$, with the variance of this asymptotic distribution reducing
		to the constant 2.

		Armed with the above asymptotic relationship and assuming that this asymptotic property approximately holds in a setting with a finite $n$, we have that
		\begin{equation*}
			n\log(\hat{\sigma}^2) \; \dot\sim \; N \left( n\log(\sigma^2), 2n \right) .
		\end{equation*}
		Thus, assuming that the model is appropriately specified, the variance of $n\log(\hat{\sigma}^2)$ will be approximately $2n$. Applying this variance back to the goodness-of-fit term,
		we see that the approximation
		\begin{equation*}
			Var \left[ -2 \ell (\hat{\theta}  ) \right] \approx 2n
		\end{equation*}
		is justified. Furthermore, $2n$ could also serve as an approximation to the variance of AIC, BIC, or AICc for this correctly specified linear regression model. This approximation
		is extremely easy to compute, and has the same form no matter the complexity of the design matrix $X$ or value of the true parameters $\theta$, making it useful as a
		general tool.

		It should be noted that this approximation does rely on asymptotic properties. In the Appendix, an exact variance for $-2 \ell (\hat{\theta})$ is found which does
		not rely on asymptotic properties. However, this variance is more complicated to compute than the simple approximation $2n$, and was not found to provide any meaningful
		benefit over $2n$ when used in procedures developed later in this chapter. Nevertheless, the derivation of this variance is presented for completeness.
		
		\section{Sandwich Estimator for Variance of Log-Likelihood GOF Term} \label{sec:sand_var}

		The preceding section developed an approximate variance for the term $-2 \ell (\hat{\theta})$ using asymptotic properties of maximum likelihood estimators under the assumption
		that a normal linear regression model is properly specified. This section will develop an estimator for the variance of $n\log(\hat{\sigma}^2)$, and thus $-2 \ell (\hat{\theta})$,
		that need not assume a given fitted normal linear regression model is correctly specified.

		Assume that we once again fit a linear regression model with assumed parameters $\theta' = [\beta', \sigma^2]$ to our data, and suppose we do not know whether this model is correctly specified.
		We wish to construct an estimator for $Var \left[ -2 \ell (\hat{\theta}  ) \right]$. This estimator will be constructed using the White robust sandwich variance estimator, employing
		much of the notation related to this development that was introduced in Chapter 2 (White, 1980).

		We let $I_{n} (\theta)$ denote the observed information matrix with regards to our specified linear regression model. With $X$ denoting the $n$ by $r$ design matrix, we see that
		the quantity $A_n (\theta)$ used in the White estimator is found to be 
		\begin{equation*}
			A_n(\theta) = \frac{1}{n}
			\begin{bmatrix}
				\frac{X'X}{\sigma^2} & \left[ \frac{-(y-X\beta)'X}{\sigma^4} \right]' \\
				\left[ \frac{-(y-X\beta)'X}{\sigma^4} \right] &  \frac{n}{2 \sigma^4} - \frac{(y-X\beta)'(y-X\beta)}{\sigma^6}
				\end{bmatrix}
				= -\frac{1}{n} I_n(\theta) .
		\end{equation*}
		Now let $x_i$ be the $i$th row of the design matrix $X$, and $y_i$ be the $i$th observation of the observation vector $y$. With these constructs at hand, we can define the score components
		for individual observations in our sample as
		\begin{equation*}
			U_i(\theta) = 
			\begin{bmatrix}
				\frac{(y_i-x_i \beta)x_i'}{\sigma^2} \\
				\frac{-1}{2 \sigma^2} + \frac{(y_i - x_i \beta)^2}{2 \sigma^4}
			\end{bmatrix}
			.
		\end{equation*}
		With these score components defined, we can then use the preceding to represent the matrix $B_n (\theta)$ used in the White estimator as
		\begin{equation*}
			B_n(\theta) = \frac{1}{n} \sum_{i=1}^{n} U_i(\theta) U_i(\theta)' .
		\end{equation*}
		Note that this matrix is an estimator for the expectation of the outer product of the score vector.

		Thus, with $B_n(\theta)$ and $A_n(\theta)$ defined, these matrices can be evaluated at the maximum likelihood estimator $\hat{\theta}$ and be used to define
		\begin{equation*}
			\begin{split}
				C_n(\hat{\theta}) & = A^{-1}_n(\hat{\theta}) B_n(\hat{\theta}) A^{-1}_n(\hat{\theta}) \\
				& = n I_n^{-1}(\hat{\theta}) [\sum_{i=1}^{n} U_i(\hat{\theta}) U_i(\hat{\theta})'] I_n^{-1}(\hat{\theta}) ,
			\end{split}
		\end{equation*}
		where $C_n(\hat{\theta})$ will be a $r+1$ by $r+1$ matrix that can be used as an estimator of the asymptotic variance-covariance matrix of $\hat{\theta}$. This estimator is robust to misspecification
		of the model. Consider the bottom rightmost element of this matrix, that being the $(r+1)^{th}$ element of the $(r+1)^{th}$ column of the matrix. This element will correspond to the large-sample
		robust variance of $\hat{\sigma}^2$. Let $s(\theta)$ refer to this corresponding element in the case of the theoretical matrix $C(\theta)$, and $s_n(\hat{\theta})$ refer to this
		corresponding element of the estimator $C_n(\hat{\theta})$. By White's results, it can then be seen that
		\begin{equation}
			\sqrt{n} (\hat{\sigma}^2 - \sigma_*^2) \xrightarrow[]{d} N(0, s(\theta_*)) ,
		\end{equation}
		where $\sigma_*^2$ denotes the pseudo-true parameter related to $\sigma^2$ in the case of potential misspecification.

		We will again employ the delta method to transform the asymptotic distribution to a form that is more suitable. We once more define a transformation of
		\begin{equation*}
			g(x) = \log(x) ,
		\end{equation*}
		and note that this function has a first derivative of
		\begin{equation*}
			g'(x) = \frac{1}{x} .
		\end{equation*}
		Applying this transformation to the asymptotic distribution presented in (4.2), we arrive at the relation
		\begin{equation*}
			\sqrt{n} ( \log (\hat{\sigma}^2) - \log(\sigma_*^2)) \xrightarrow[]{d} N \left( 0, \frac{1}{\sigma_*^4} s(\theta_*) \right) .
		\end{equation*}
		Using this asymptotic distribution, one can arrive at an approximate distribution for $n\log(\hat{\sigma}^2)$ as
		\begin{equation*}
			n\log(\hat{\sigma}^2) \; \dot\sim \; N \left( n\log(\sigma_* ^2), \frac{n}{\sigma_*^4} s(\theta_*) \right) ,
		\end{equation*}
		which could be suitable for use in finite sample sizes that are sufficiently large. However, $\sigma_*^2$ and $s(\theta_*)$ are likely to be unknown in practical modeling
		applications. Thus, estimating these quantities with $\hat{\sigma}^2$ and $s_n(\hat{\theta})$ respectively, a reasonable estimate for the variance of $n\log(\hat{\sigma}^2)$ can
		be found to be
		\begin{equation*}
			Var \left[ n\log(\hat{\sigma}^2) \right] \approx \frac{n}{\hat{\sigma}^4} s_n(\hat{\theta}) .
		\end{equation*}
		By the relation presented in (4.1) in the previous section, it is clear that this variance estimate is also suitable for $Var \left[ -2 \ell (\hat{\theta}  ) \right]$, and therefore
		likelihood-based information criteria as a whole that possess a constant penalty term. This estimator should converge to our previously derived value $2n$ in the case of a correctly
		specified model, as the sandwich estimator component will converge to the expected Fisher information used in section 4.1, and the MLE $\hat{\sigma}^2$ should converge to the true
		parameter value $\sigma^2$. However, this sandwich estimator need not assume correct specification, and should be relatively robust to model misspecification.
		
		For the remainder of this thesis, this sandwich estimator will be referred to as $\widehat{Var}[GOF]$, such that
		\begin{equation*}
			\widehat{Var}[GOF] = \frac{n}{\hat{\sigma}^4} s_n(\hat{\theta}) .
		\end{equation*}
		This notation will be used because this statistic is estimating $Var[GOF]$, the true variance of the likelihood goodness-of-fit
		term.
		
		\section{Bootstrap Test for GOF of a Linear Model} \label{sec:boot_test}

		The previous sections have established an asymptotic variance for the likelihood goodness-of-fit term in the case of a correctly specified normal linear regression model, and an
		estimator for this variance that does not assume the model is correctly specified. We will now synthesize these two developments to form a general goodness-of-fit procedure to
		test the hypothesis that a given normal linear regression model is correctly specified.

		Under the null hypothesis that a normal linear regression model is correctly specified, the estimator $\widehat{Var}[GOF]$ should be close to the theoretical value $2n$ for a
		sufficient sample size. Thus, we need some way to probabilistically determine whether a fitted model demonstrates this property that indicates correct specification, or whether it
		appears to violate the null hypothesis.
		
		We propose the use of the non-parametric bootstrap to obtain an empirical estimate for the sampling distribution of $\widehat{Var}[GOF]$. Once this empirical distribution
		has been obtained, the null hypothesis can be tested by observing whether a bootstrap interval for $Var[GOF]$ contains the theoretical value $2n$, as the approximation
		$Var[GOF] \approx 2n$ should hold for sufficient sample sizes under the null hypothesis. If a $100*(1-\alpha)$\% bootstrap confidence interval does not contain 
		$2n$, we reject the null hypothesis and conclude that the model is misspecified; however, if the interval does contain $2n$, we do not have sufficient evidence to reject
		the null hypothesis, and the proposed model does not demonstrate lack-of-fit. Note that in the case of a model that is merely overspecified, the null hypothesis is technically
		true as the model contains requisite structure.

		A full summary of the proposed procedure can be found below.
		\begin{algorithm}
			\caption{Bootstrap Goodness-of-Fit Test for a Normal Linear Regression Model}
			\begin{algorithmic}[1]
			  \Statex For test level $\alpha$, candidate normal linear model $M$, bootstrap iterations $B$, sample size $n$, and a null hypothesis that $M$
			  is correctly specified or overspecified:
			  \State Resample, with replacement, outcomes with covariates to generate a bootstrap sample of size $n$.
			  \State Fit model $M$ to this bootstrap sample, and with this fitted model, calculate $\widehat{Var}[GOF]$
			  and record this statistic.
			  \State Repeat steps 1-2 $B$ times to generate an empirical bootstrap distribution for $\widehat{Var}[GOF]$.
			  \State Construct a $100*(1-\alpha)$\% bootstrap confidence interval for $Var[GOF]$.
			  \State If this interval does not contain $2n$, reject the null hypothesis at the $\alpha$ level. If it does contain
			  $2n$, the null hypothesis was not rejected and model $M$ does not appear to exhibit lack-of-fit. 
			\end{algorithmic}
		\end{algorithm}

		This procedure can be used to assess the general hypothesis that a normal linear regression model is properly specified against the alternative that it displays lack-of-fit.
		Unlike other goodness-of-fit tests and procedures, this method does not test a specific assumption of linear regression such as normality or homoskedasticity, but rather
		all assumptions of normal linear regression. If any of them are violated, the property $Var[GOF] \approx 2n$ is not guaranteed to hold. This characteristic allows the test to detect
		many forms of misspecification from mean misspecification to distributional misspecification to heteroskedasticity.
		
		The following section will demonstrate the efficacy of this procedure to roughly maintain the desired size when the null hypothesis is true. Further simulations will display power
		of the test to reject the null hypothesis when it is false in a variety of different scenarios. However, it should be noted that this test will come with added computational
		complexity over other goodness-of-fit tests due to the need to perform the bootstrap.
		
		While the following simulations will employ a percentile interval as the bootstrap confidence interval method of choice, one is not limited to using this method
		and may use any bootstrap interval they choose so long as it is theoretically justifiable. Additionally, the non-parametric bootstrap is employed in the algorithm detailed above to avoid further
		assumptions. Limited testing results suggest that the residual bootstrap may also work just as well in this procedure. However, pending further investigation, the non-parametric
		bootstrap is recommended at the current time. All simulations presented in this thesis will use the non-parametric bootstrap flavor of the procedure.
		
		\section{Comparative Simulations Across Different GOF Procedures} \label{sec:gof_sim}

		The following subsections detail simulation studies performed using the goodness-of-fit procedure presented above. The first subsection details a scenario where the null
		hypothesis of the test is true, and demonstrates that the proposed bootstrap test can approximately maintain the desired size. The second subsection 
		explores the power of the test in the case of mean misspecification, and the third subsection explores the power of the test in the case of heteroskedasticity induced by
		unobserved covariates.

		\subsection{Simulation to Establish Type~I Error Rate of Bootstrap GOF Test}

		The data generating process for this simulation is to be as follows. The $n$ by 1 outcome vector $y = [y_1,...,y_n]'$ will be generated for $i = 1,...,n$ according
		to
		\begin{equation*}
			y_i = 2.0 + 2.0 x_{i1} + 2.0 x_{i2} + \epsilon_i , 
		\end{equation*}
		where $\epsilon_i \stackrel{iid}{\sim} N(0,4)$ is a randomly generated normal error, and $x_{i1}$ and $x_{i2}$ are completely $iid$ covariates generated according to
		a $Uniform(0,5)$ distribution. Note this formulation is one of a normal linear regression model with an intercept and an effect for each covariate present.

		With the generated data in hand, we will proceed to fit a normal linear regression model that includes an intercept and effects for $x_{i1}$ and $x_{i2}$. This model
		is properly specified, and should not exhibit gross lack-of-fit. For each iteration of the simulation, the bootstrap goodness-of-fit test presented above will be
		performed on the fitted model at the $\alpha = .05$ level, and whether the test rejects the null hypothesis will be recorded. In this way, after many simulation iterations
		we will be able to estimate the Type~I error rate of the test for this scenario as the null hypothesis of the test is true.

		Additionally, the White test (White, 1980) and the Breusch-Pagan test (Breusch and Pagan, 1979) will also be performed on the fitted model in each simulation iteration, and
		whether these tests reject their null hypotheses of homoskedasticity will be recorded. This is done in order to establish the ability of these tests to maintain a desired Type
		I error rate for future simulations.

		The simulation will be performed for $n = 100,200,...,1000$. The number of bootstrap iterations used by the bootstrap goodness-of-fit test will be $B = 1000$ for all simulations.
		This simulation will be performed $1000$ times for each value of $n$, with the Type~I error rate to be calculated afterwards. A full summary of the simulation procedure can be found below.
		\begin{algorithm}[H]
			\caption*{\textbf{Simulation 3} Type~I Error Rate Simulation, Normal Linear Regression Goodness-of-Fit Tests}
			\begin{algorithmic}[1]
			  \Statex For $n = 100,200,...,1000$, $B = 1000$, and $\alpha = 0.05$, do the following:
			  \State Draw a test sample of size $n$ using the specified true model.
			  \State Fit the candidate model to the data.
			  \State Perform the bootstrap goodness-of-fit test, White test, and Breusch-Pagan test at the $\alpha$ level,
			  and record whether each test rejects its null hypothesis.
			  \State Repeat steps 1-3 1000 times.
			  \State Calculate the Type~I error rate of each test.
			\end{algorithmic}
		\end{algorithm}

		This simulation was performed in Julia, version 1.8.1 (Bezanson et al., 2017). The results of the simulation can be found in the table below.

		\begin{table}[H]
			\centering
			\small\addtolength{\tabcolsep}{-3pt}
			\setlength\extrarowheight{-3pt}
			\ttabbox[\FBwidth]
			{\caption{\label{tab:sim3_table}Simulation 3 Results - Type~I Error}}
			{
			\begin{tabular}{ c|c|c|c}
			$n$ & \vtop{\hbox{\strut Bootstrap}\hbox{\strut Test}} & \vtop{\hbox{\strut White}\hbox{\strut Test}} & \vtop{\hbox{\strut Breusch-Pagan} \hbox{\strut Test}} \\
			 \hline
			 100 & 0.088 & 0.043 & 0.045 \\
			 200 & 0.100 & 0.054 & 0.056 \\
			 300 & 0.089 & 0.051 & 0.410 \\
			 400 & 0.082 & 0.061 & 0.043 \\
			 500 & 0.086 & 0.045 & 0.047 \\
			 600 & 0.076 & 0.053 & 0.051 \\
			 700 & 0.079 & 0.039 & 0.046 \\
			 800 & 0.082 & 0.060 & 0.045 \\
			 900 & 0.082 & 0.030 & 0.040 \\
			 1000 & 0.067 & 0.055 & 0.064 \\
			 \Xhline{3\arrayrulewidth}
			\end{tabular}
			}
		\end{table}

		The bootstrap test exhibits slight anti-conservative behavior across all values of $n$ with the empirically determined Type~I error rate exceeding the desired Type~I error rate
		of $\alpha = 0.05$ in all scenarios. However, as $n$ increases, the empirical Type~I error rate appears to be decreasing towards the desired value. Additionally, the
		behavior of the test is consistently anti-conservative for the scenarios shown. Thus, when using the procedure, one may keep in mind that the test may be more prone to
		rejection than one might expect, particularly for small sample sizes before asymptotic properties of the bootstrap truly manifest.

		The White and Breusch-Pagan tests roughly maintain the desired Type~I error rate. However, there is some variation in the behavior displayed by the table, as some values
		of $n$ display conservative test behavior while others display anti-conservative behavior. Thus, while the bootstrap test appears to trend from anti-conservative behavior
		to the desired behavior as $n$ increases, the White and Breusch-Pagan test behavior in finite sample sizes cannot exactly be determined.

		\subsection{Simulation to Establish Power of Bootstrap GOF Test - Mean Misspecification}

		The previous simulation considered a scenario where the null hypothesis of all tests were true. This simulation will consider a scenario where the White and
		Breusch-Pagan test null hypotheses are true, but where the general hypothesis of the bootstrap test that a normal linear regression model is properly
		specified is false.

		The data generating process will be the exact same as the previous simulation, where the $n$ by 1 outcome vector $y = [y_1,...,y_n]'$ will be generated
		for $i = 1,...,n$ according to
		\begin{equation*}
			y_i = 2.0 + 2.0 x_{i1} + 2.0 x_{i2} + \epsilon_i , 
		\end{equation*}
		where $\epsilon_i \stackrel{iid}{\sim} N(0,4)$ is a randomly generated normal error, and $x_{i1}$ and $x_{i2}$ are completely $iid$ covariates generated according to
		a $Uniform(0,5)$ distribution. However, in this case the fitted model will be a normal linear regression model with an intercept and effect for only
		$x_{i1}$. The covariate $x_{i2}$ will be omitted from the model, and could be viewed as a characteristic that affects the true generating process, but
		was unobserved in data collection. This omission will induce mean misspecification into this fitted model as the mean structure will be underspecified.

		For each simulation iteration, the bootstrap goodness-of-fit test, White test, and Breusch-Pagan test will each be performed at the $\alpha = .05$ level on the 
		fitted model. The White and Breusch-Pagan tests do not have their null hypotheses violated as the data still exhibits homoskedasticity, and thus they should not be able to
		detect this form of misspecification. We expect to tests to only maintain their Type~I error rates that were observed in the previous simulation. However, the bootstrap test has its null hypothesis violated,
		and thus the power of the test in this scenario will be determined. After many simulation iterations, the Type~I error rates or power of each test
		will be calculated based on the proportion of times each test rejected its null hypothesis.

		The simulation will again be performed for $n = 100,200,...,1000$, with $B = 1000$ for all simulations. The simulation will be performed $1000$ times for each value of $n$, with
		Type~I error rate and power to be calculated afterwards. A full summary of the simulation procedure can be found below.
		\begin{algorithm}[H]
			\caption*{\textbf{Simulation 4} Type~I Error Rate and Power Simulation, Mean Misspecification, Normal Linear Regression Goodness-of-Fit Tests}
			\begin{algorithmic}[1]
			  \Statex For $n = 100,200,...,1000$, $B = 1000$, and $\alpha = 0.05$, do the following:
			  \State Draw a test sample of size $n$ using the specified true model.
			  \State Fit the underspecified candidate model to the data.
			  \State Perform the bootstrap goodness-of-fit test, White test, and Breusch-Pagan test at the $\alpha$ level,
			  and record whether each test rejects its null hypothesis.
			  \State Repeat steps 1-3 1000 times.
			  \State Calculate the Type~I error rate or power of each test.
			\end{algorithmic}
		\end{algorithm}

		This simulation was performed in Julia, version 1.8.1 (Bezanson et al., 2017). The results of the simulation can be found in the table below, with the column for the
		bootstrap test displaying power and the columns for the White and Breusch-Pagan tests displaying Type~I error rate.

		\begin{table}[H]
			\centering
			\small\addtolength{\tabcolsep}{-3pt}
			\setlength\extrarowheight{-3pt}
			\ttabbox[\FBwidth]
			{\caption{\label{tab:sim4_table}Simulation 4 Results - Power and Type~I Error}}
			{
			\begin{tabular}{ c|c|c|c}
			$n$ & \vtop{\hbox{\strut Bootstrap}\hbox{\strut Test}} & \vtop{\hbox{\strut White}\hbox{\strut Test}} & \vtop{\hbox{\strut Breusch-Pagan} \hbox{\strut Test}} \\
			 \hline
			 100 & 0.516 & 0.056 & 0.038 \\
			 200 & 0.735 & 0.056 & 0.063 \\
			 300 & 0.872 & 0.044 & 0.050 \\
			 400 & 0.925 & 0.041 & 0.055 \\
			 500 & 0.963 & 0.042 & 0.049 \\
			 600 & 0.986 & 0.043 & 0.047 \\
			 700 & 0.987 & 0.064 & 0.058 \\
			 800 & 0.994 & 0.051 & 0.037 \\
			 900 & 0.998 & 0.049 & 0.055 \\
			 1000 & 1.000 & 0.049 & 0.042 \\
			 
			 \Xhline{3\arrayrulewidth}
			\end{tabular}
			}
		\end{table}

		The power of the bootstrap test in this scenario of mean misspecification is rather modest for low values of $n$ displayed in the table. However, power rises quickly to high levels
		with the test appearing to possess near-perfect power as the value of $n$ approaches 1000. Thus, the bootstrap test appears to possess the ability to detect this mean
		misspecification on account of an unobserved covariate, and the power to detect this misspecification rises as the sample size increases as one would expect.

		In contrast, the White test and Breusch-Pagan test both seem to only maintain their Type~I error rates established in Simulation 3. This performance was to be expected
		as the null hypotheses of these tests are not violated in this scenario. However, this simulation setting does demonstrate how the inability of a goodness-of-fit test to detect lack of
		fit does not mean lack-of-fit is not present, as these two tests are limited in the scope of misspecification they are able to detect. This limitation in scope is in contrast to the
		bootstrap test, which possesses a much more general null hypothesis.

		\subsection{Simulation to Establish Power of Bootstrap GOF Test - Heteroskedasticity Induced by Unobserved Covariates}

		The final simulation of this section will explore a scenario where the true generating model exhibits heteroskedasticity courtesy of a variable that will be
		treated as unobserved. In this setting, the null hypothesis of the bootstrap test, the White test, and the Breusch-Pagan test will all be violated, and the
		power of each will be compared.

		The data generating process for this simulation is to be as follows. The $n$ by 1 outcome vector $y = [y_1,...,y_n]'$ will be generated for $i = 1,...,n$ according
		to
		\begin{equation*}
			y_i = 2.0 + 2.0 x_{i1} + 2.0 x_{i2} + \epsilon_i , 
		\end{equation*}
		where $\epsilon_i \stackrel{iid}{\sim} N \left( 0,(2 + x_{i3})^2 \right)$ is a randomly generated normal error, and $x_{i1}$, $x_{i2}$, and $x_{i3}$ are completely $iid$ covariates
		generated according to a $Uniform(0,5)$ distribution. Note that heteroskedasticity is introduced into this model courtesy of $x_{i3}$ affecting the error variance, and thus any
		linear model fit to the data will not be properly specified if it assumes homoskedasticity.
		
		The fitted model for this simulation will be a normal linear regression model that has the correct mean structure of an intercept with effects for $x_{i1}$ and $x_{i2}$, but will
		be fit assuming homoskedasticity and therefore be misspecified. The covariate $x_{i3}$ is assumed to be unobserved in this modeling scenario as if it were not collected, and thus the
		source of heteroskedasticity in the data is not visible to the statistician. The following figure displays an example residual plot using a dataset drawn from the generating distribution
		where the proposed misspecified model is used.

		\begin{figure}[H]
			\centering
			\captionsetup{justification=centering}
			\includegraphics[width=1\textwidth]{figures/sim5_rsidual_plot.pdf}
			\caption{\label{fig:sim5_residual_plot} Simulation 5 Example Fitted Model Residual Plot}
		\end{figure}

		As the heteroskedasticity in the generating model is induced by a covariate that was not observed or used in the model in any way, we do not see any visible departures from
		homoskedasticity in this residual plot. Thus, one would not by visual inspection ascertain that the proposed linear model is misspecified. 

		For each simulation iteration, the bootstrap test, White test, and Breusch-Pagan test will be performed on the given fitted model at the $\alpha = 0.05$. After all iterations have been
		performed, the power of each test will be calculated, as there is a violation of the null hypothesis of each test. The ability of each test to detect this misspecification will
		then be assessed.

		The simulation will also be performed for $n = 100,200,...,1000$, with $B = 1000$ for all values of $n$, and $1000$ simulation iterations for each value of $n$. 
		Power will be calculated after all iterations have been completed for a given $n$. A full summary of the simulation procedure can be found below.
		\begin{algorithm}[H]
			\caption*{\textbf{Simulation 5} Power Simulation, Heteroskedasticity, Normal Linear Regression Goodness-of-Fit Tests}
			\begin{algorithmic}[1]
			  \Statex For $n = 100,200,...,1000$, $B = 1000$, and $\alpha = 0.05$, do the following:
			  \State Draw a test sample of size $n$ using the specified true model.
			  \State Fit the underspecified candidate model to the data.
			  \State Perform the bootstrap goodness-of-fit test, White test, and Breusch-Pagan test at the $\alpha$ level,
			  and record whether each test rejects its null hypothesis.
			  \State Repeat steps 1-3 1000 times.
			  \State Calculate the power of each test.
			\end{algorithmic}
		\end{algorithm}

		This simulation was performed in Julia, version 1.8.1 (Bezanson et al., 2017). The results of the simulation are summarized in the table below.

		\begin{table}[H]
			\centering
			\small\addtolength{\tabcolsep}{-3pt}
			\setlength\extrarowheight{-3pt}
			\ttabbox[\FBwidth]
			{\caption{\label{tab:sim5_table}Simulation 5 Results - Power}}
			{
			\begin{tabular}{ c|c|c|c}
			$n$ & \vtop{\hbox{\strut Bootstrap}\hbox{\strut Test}} & \vtop{\hbox{\strut White}\hbox{\strut Test}} & \vtop{\hbox{\strut Breusch-Pagan} \hbox{\strut Test}} \\
			 \hline
			 100 & 0.089 & 0.059 & 0.059 \\
			 200 & 0.410 & 0.051 & 0.054 \\
			 300 & 0.616 & 0.045 & 0.046 \\
			 400 & 0.755 & 0.048 & 0.044 \\
			 500 & 0.890 & 0.055 & 0.059 \\
			 600 & 0.927 & 0.050 & 0.047 \\
			 700 & 0.969 & 0.050 & 0.056 \\
			 800 & 0.984 & 0.050 & 0.059 \\
			 900 & 0.994 & 0.048 & 0.052 \\
			 1000 & 0.998 & 0.044 & 0.050 \\
			 
			 \Xhline{3\arrayrulewidth}
			\end{tabular}
			}
		\end{table}

		The bootstrap test seems to only exhibit power in line with its empirical Type~I error rate displayed in Simulation 3 for $n = 100$, but the power of the test rises
		rapidly, reaching near-perfect levels by the time $n = 1000$. In contrast, the White test and Breusch-Pagan test exhibit only power in line with their Type~I error
		rates established in Simulation 3, and thus do not seem to be able to detect that the model is misspecified despite the misspecification being a form of
		heteroskedasticity. This limitation is on account of how these tests are performed, as they rely on looking for heteroskedasticity that is induced by observed covariates.
		When heteroskedasticity is induced by an unobserved covariate that is not used in the model, the White test and Breusch-Pagan test cannot ascertain that their
		null hypotheses of homoskedasticity are violated. However, the bootstrap test is able to reject its more general null hypothesis.

		If one were to perform a simulation wherein heteroskedasticity is induced by observed covariates, the White and Breusch-Pagan tests would exhibit higher power in lower
		sample sizes than the bootstrap test, as these two tests are specifically tuned for this type of misspecification. Thus, the bootstrap test in a sense trades power in specific
		scenarios for generality, as it is able to detect many more forms of misspecification.
		
		This characteristic makes the bootstrap test suitable as an omnibus goodness-of-fit procedure for normal linear regression models as it is able to detect of myriad of different
		forms of misspecification, unlike the other tests observed in the above simulations. However, rejection of the bootstrap test does not explicitly inform one as to why the test
		was rejected, and further investigation will be needed to determine the nature of the misspecification and where one should proceed next in a model selection endeavor.



